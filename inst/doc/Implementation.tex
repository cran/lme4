\documentclass[12pt]{article}
\usepackage{Sweave}
\usepackage{myVignette}
\usepackage[authoryear,round]{natbib}
\bibliographystyle{plainnat}
\DefineVerbatimEnvironment{Sinput}{Verbatim}
{formatcom={\vspace{-2.5ex}},fontshape=sl,
  fontfamily=courier,fontseries=b, fontsize=\scriptsize}
\DefineVerbatimEnvironment{Soutput}{Verbatim}
{formatcom={\vspace{-2.5ex}},fontfamily=courier,fontseries=b,%
  fontsize=\scriptsize}
%%\VignetteIndexEntry{Implementation Details}
%%\VignetteDepends{Matrix}
%%\VignetteDepends{lme4}
\begin{document}


\setkeys{Gin}{width=\textwidth}
\title{Linear mixed model implementation in lme4}
\author{Douglas Bates\\Department of Statistics\\University of
  Wisconsin -- Madison\\\email{Bates@wisc.edu}}
\maketitle
\begin{abstract}
  Expressions for the evaluation of the profiled log-likelihood or
  profiled log-restricted-likelihood of a linear mixed model, the
  gradients and Hessians of these criteria, and update steps for an
  ECME algorithm to optimize these criteria are given in Bates and
  DebRoy (2004).  A representation of linear mixed models using
  positive semidefinite symmetric matrices and dense matrices is given
  in Bates (2004).  In this paper we present details of that
  representation and those computational methods in the \code{lme4}
  package for R.
\end{abstract}

